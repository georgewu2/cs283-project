\documentclass[12pt]{article}
\usepackage{geometry}    
\usepackage{moreverb}   
\usepackage{fancyhdr}
\geometry{letterpaper}
\usepackage{algorithmic}             
\usepackage{algorithm}
\usepackage{array}     
\usepackage{hyperref}
\usepackage{graphicx}
\usepackage{fullpage}	
\usepackage{amsmath, amssymb, amsthm}
\usepackage{mathabx}
\usepackage{float}
\usepackage{bm}	
\graphicspath{ {data/}  {data/results} }
\usepackage{url}


\begin{document}

\title{CS 283 Final Project \\ Poisson Image Editing}
\date{Fall 2014}
\author{George Wu}

\maketitle

%%%%%%%%%%%%%%%%%%%%%%%%%%%%%%%%%%%%%%%%%%%%%%%%%%%%%
\section{Introduction}
What if you could seamlessly blend parts of images to make it seem as if those parts actually transpired in the same setting? As photography has shifted to the digital world, photo editing has become more accessible to the average consumer, with software such as Adobe Photoshop \cite{photoshop} and Lightroom \cite{lightroom} allowing such capabilities. Traditional methods involve directly copying and pasting portions of a source image's contents onto a background and blending the edges, but unless the source and background are already similar, these edges will still be visible. We will explore ways to improve this process, borrowing heavily from the Poisson equation

%%%%%%%%%%%%%%%%%%%%%%%%%%%%%%%%%%%%%%%%%%%%%%%%%%%%%
\section{Methods}

%%%%%%%%%%%%%%%%%%%%%%%%%%%%%%%%%%%%%%%%%%%%%%%%%%%%%
\section{Results} 

%%%%%%%%%%%%%%%%%%%%%%%%%%%%%%%%%%%%%%%%%%%%%%%%%%%%%
\section{Conclusion}

%%%%%%%%%%%%%%%%%%%%%%%%%%%%%%%%%%%%%%%%%%%%%%%%%%%%%
\section{References}
\begin{thebibliography}{99}

\bibitem{lightroom}
Adobe Lightroom. Adobe, Inc. \textless\url{https://www.adobe.com/products/photoshop-lightroom.html?promoid=KLXLX}\textgreater

\bibitem{photoshop}
Adobe Photoshop. Adobe, Inc. \textless\url{http://www.adobe.com/products/photoshop.html}\textgreater.
\end{thebibliography}


\end{document}